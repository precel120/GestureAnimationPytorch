\documentclass[12pt]{article}
\usepackage[utf8]{inputenc}
\usepackage{color}
\usepackage{polski}
\usepackage{graphicx}
\usepackage{pdfpages}
\usepackage{indentfirst}
\usepackage{geometry}
\usepackage{setspace}
\usepackage[hyphens]{url}
\usepackage[style=numeric, backend=biber]{biblatex}
\usepackage{array}
\usepackage{float}
\usepackage{listings}

\addbibresource{bibliografia.bib}
\graphicspath{{./zdjecia/}}

\renewcommand\lstlistlistingname{Listingi}

\setstretch{1.5}

\geometry{
 a4paper,
 left=2.5cm,
 top=1.25cm,
 right=2.5cm
}

\begin{document}
\begin{sloppypar}
\includepdf{Strona_Tytulowa.pdf}

\tableofcontents
\newpage

\section{Wprowadzenie}
{
  \subsection{Problematyka}
  {
    W dzisiejszych czasach bardzo modnym tematem jest sztuczna inteligencja, która zaczyna się wkradać w każdy aspekt naszego życia.
    Możemy ją spotkać w formie chatów, podpowiedzi do pisanego kodu, asystentów internetowych, systemów rozpoznawania głosów, czy nawet we własnej lodówce!
    Każda firma żeby zaistnieć i pozostać istotną inwestuje w tę część technologii. Jednakże to z czym AI radzi sobie najgorzej są ręce.
    \newline
    W tym projekcie chodzi o stworzenie sieci typu GAN, która pozwoli na generowanie obrazów rąk, w jak najlepszej jakości.
    Dodatkowo sieć ma za zadanie nauczyć się, żeby móc modyfikować istniejące zdjęcia i nadawać im zupełnie inny gest, przy zachowaniu jakości i realizmu.
    Szczególnie ten drugi aspekt pozostaje dla sztucznej inteligencji problematyczny. 
    Myślę, że każdy z nas spotkał się ze zdjęciami, które dosłownie wyglądają, jak żywe, ale to co najczęściej zdradza, że jednak to AI maczało w nim palce są ręce.
    Za długie palce, dziwne ich ułożenie, ilość, czy nawet totalnie odrealniony wygląd. 
    Przeróżne firmy, jak i naukowcy stale ulepszają sieci, i rozwiązania, żeby i to przestało być problemem.
    Niniejsza praca również podejmuje się tego niełatwego zadania.
  }
  \subsection{Cel i zakres pracy}
  {
    Celem niniejszej pracy jest stworzenie sieci neuronowej typu GAN, 
    która pozwoli na generowanie obrazów gestów rąk, w jak najlepszej jakości.
    \newline
    Docelowo również, wygenerowane zdjęcia będą wykorzystywane do stworzenia animacji przechodzenia z jednego gestu w inny.
  }
  \subsection{Struktura pracy}
  {
    Pierwszy rozdział przybliży to czym są sieci neuronowe, a dokładniej typu GAN, jakie są analogiczne rozwiązania, oraz o samym generowaniu zdjęć.
    Następny opowie jakie narzędzia, biblioteki i technologie zostały wykorzystane do realizacji projektu. 
    Trzeci zaś mówi o tym jak wyglądał proces tworzenia projektu. Co po koleji zostało zrobione, jakie po drodze wystąpiły komplikacje, oraz jak zostały rozwiązane i finalnie jak wygląda projekt.
    Przedostatni rozdział to przedstawienie wyników, rezultatów realizowanego projektu, analiza i omówienie ich.
    Ostatni rozdział zawiera wnioski końcowe i podsumowanie.
  }
}

\section{Sieci GAN, analiza konkurencji i techniczne aspekty realizacji}
{
  \subsection{Sieci neuronowe i AI}
  {
    sieci
  }
  \subsection{Sieci typu GAN}
  {
    GAN
  }
  \subsection{Generowanie obrazów rąk}
  {
    generowanie
  }
  \subsection{GestureGAN}
  {
    GestureGAN
  }
  \subsection{PoseGAN}
  {
    PoseGAN
  }
}

\section{Narzędzia i technologie wybrane do realizacji projektu}
{
  \subsection{Python}
  {
    Językiem programowania wykorzystanym do realizacji projektu jest Python. 
    Jest to bardzo oczywisty wybór, ponieważ jest to najpopularniejszy język programowania do tworzenia sieci neuronowych w dzisiejszych czasach.
    Przybliżmy go jednak, według oficjalnej dokumentacji, jest "łatwym do nauczenia się i potężnym językiem programowania. 
    Posiada wydajne struktury danych wysokiego poziomu oraz proste, ale skuteczne podejście do programowania obiektowego"\cite{'python-docs'}.
    Co niejako wyróżnia go na tle innych języków to fakt, że jest dynamicznie typowany oraz jest językiem interpetowanym. 
    To oznacza, że nie posiada kompilatora a interpeter, który nie kompiluje programu do pliku wykonywalnego, a kod jest wykonywany w czasie rzeczywistym.
    Czyni to Python językiem łatwym w testowaniu, kompilowaniu czy wykonywaniu, przenośny, co oznacza, że ten sam kod uruchomi się niezależnie od systemu operacyjnego czy urządzenia.
    Sam język Python jest zbudowany na bibliotece C, co oznacza, że jest bardzo szybki i wydajny. 
    Został stworzony przez Guido van Rossuma w roku 1991.
    Co ciekawe jego nazwa wywodzi się z starych serii skeczów grupy Monty Python’s Flying Circus, 
    a zyskał na popularności w momencie gdy firma Google, powiedziała, że wykorzystuje go do własnych, wewnętrzych celów.
    \newline
    Jednakże czemu akurat to on jest najczęściej wykorzystany do AI? Odpowiedź jest tak prosta jak sam Python jest prosty.
    Wynika to z tego, że Python ma bardzo prostą i czytelną składnię, co pozwala developerom skupianie się na logice i samym problemie, a nie na składni\cite{'python-ai'}.
    Dodatkowo Python sam w sobie nie wymaga dużej ilości kodu. Najprostsza sieć neuronowa może zostać stworzona i uruchomiona w zaledwie 4 linijki!
    Kolejnym powodem przemawiającym dlaczego to własnie Python jest najczęściej używany, 
    jest bardzo duża ilość bibliotek, czy to wbudowanych, czy stworzonych przez społeczność. 
    Biblioteki takie jak NumPy, SciPy, Matplotlib, czy wykorzystane w tym projekcie PyTorch czy MediaPipe, o których będzie w kolejnych podroździałach.
    Także kolejnym i osatnim już aspektem, o którym chcę wspomnieć, jest wyżej wymieniona społeczność. 
    To dzięki dużej liczbie osób i ogromnym zebranym doświadczeniu, tworzenie dowolnego projektu staje się znacznie prostsze. 
    Praktycznie każdy projekt, aspekt projektu czy problem natrafiało wcześniej duża część ludzi, dzięki czemu możemy szybko i sprawnie rozwiązaywać problem.
  }
  \subsection{PyTorch}
  {
    pytorch
  }
  \subsection{MediaPipe}
  {
    mediapipe
  
  }
  \subsection{OpenCV}
  {
    opencv
  }
}

\section{Proces tworzenia projektu}
{
  \subsection{Dobór zbioru danych}
  {
    zbior
  }
  \subsection{Wybór architektury sieci}
  {
    architektura
  }
  \subsection{Preprocesowanie danych}
  {
    Preprocesowanie
  }
  \subsection{Tworzenie modelu}
  {
    model
  }
  \subsection{Tesotowanie modelu}
  {
    Tesotowanie
  }
  \subsection{Realizacja projektu}
  {
    realizacja
  }
}

\section{Wyniki i dyskusja}
{
  wyniki
}

\section{Podsumowanie}
{
  \subsection{Zalety i wady przyjętych rozwiązań}
  {
    Zalety
  }
  \subsection{Napotkane trudności}
  {
    trudności
  }
  \subsection{Możliwości rozwoju}
  {
    rozwoj
  }
  \subsection{Wnioski końcowe}
  {
    wnioski
  }
}

\clearpage
\printbibliography[
  heading=bibintoc,
  title={Bibliografia}
]

\clearpage
\listoffigures

\clearpage
\listoftables

\clearpage
\addcontentsline{toc}{section}{Listingi}
\lstlistoflistings

\end{sloppypar}
\end{document}