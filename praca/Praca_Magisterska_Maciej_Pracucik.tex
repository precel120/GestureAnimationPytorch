\documentclass[12pt]{article}
\usepackage[utf8]{inputenc}
\usepackage{color}
\usepackage{polski}
\usepackage{graphicx}
\usepackage{pdfpages}
\usepackage{indentfirst}
\usepackage{geometry}
\usepackage{setspace}
\usepackage[hyphens]{url}
\usepackage[style=numeric, backend=biber]{biblatex}
\usepackage{array}
\usepackage{float}
\usepackage{listings}

\addbibresource{bibliografia.bib}
\graphicspath{{./zdjecia/}}

\renewcommand\lstlistlistingname{Listingi}

\setstretch{1.5}

\geometry{
 a4paper,
 left=2.5cm,
 top=1.25cm,
 right=2.5cm
}

\begin{document}
\begin{sloppypar}
\includepdf{Strona_Tytulowa.pdf}

\tableofcontents
\newpage

\section{Wprowadzenie}
{
  \subsection{Problematyka}
  {
    W dzisiejszych czasach bardzo modnym tematem jest sztuczna inteligencja, która zaczyna się wkradać w każdy aspekt naszego życia.
    Możemy ją spotkać w formie chatów, podpowiedzi do pisanego kodu, asystentów internetowych, systemów rozpoznawania głosów, czy nawet we własnej lodówce!
    Każda firma żeby zaistnieć i pozostać istotną inwestuje w tę część technologii. Jednakże to z czym AI radzi sobie najgorzej są ręce.
    \newline
    W tym projekcie chodzi o stworzenie sieci typu GAN, która pozwoli na generowanie obrazów rąk, w jak najlepszej jakości.
    Dodatkowo sieć ma za zadanie nauczyć się, żeby móc modyfikować istniejące zdjęcia i nadawać im zupełnie inny gest, przy zachowaniu jakości i realizmu.
    Szczególnie ten drugi aspekt pozostaje dla sztucznej inteligencji problematyczny. 
    Myślę, że każdy z nas spotkał się ze zdjęciami, które dosłownie wyglądają, jak żywe, ale to co najczęściej zdradza, że jednak to AI maczało w nim palce są ręce.
    Za długie palce, dziwne ich ułożenie, ilość, czy nawet totalnie odrealniony wygląd. 
    Przeróżne firmy, jak i naukowcy stale ulepszają sieci, i rozwiązania, żeby i to przestało być problemem.
    Niniejsza praca również podejmuje się tego niełatwego zadania.
  }
  \subsection{Cel i zakres pracy}
  {
    Celem niniejszej pracy jest stworzenie sieci neuronowej typu GAN, 
    która pozwoli na generowanie obrazów gestów rąk, w jak najlepszej jakości.
    \newline
    Docelowo również, wygenerowane zdjęcia będą wykorzystywane do stworzenia animacji przechodzenia z jednego gestu w inny.
  }
  \subsection{Struktura pracy}
  {
    Pierwszy rozdział przybliży to czym są sieci neuronowe, a dokładniej typu GAN, jakie są analogiczne rozwiązania, oraz o samym generowaniu zdjęć.
    Następny opowie jakie narzędzia, biblioteki i technologie zostały wykorzystane do realizacji projektu. 
    Trzeci zaś mówi o tym jak wyglądał proces tworzenia projektu. Co po koleji zostało zrobione, jakie po drodze wystąpiły komplikacje, oraz jak zostały rozwiązane i finalnie jak wygląda projekt.
    Przedostatni rozdział to przedstawienie wyników, rezultatów realizowanego projektu, analiza i omówienie ich.
    Ostatni rozdział zawiera wnioski końcowe i podsumowanie.
  }
}

\section{Sieci GAN, analiza konkurencji i techniczne aspekty realizacji}
{
  \subsection{Sieci neuronowe i AI}
  {
    sieci
  }
  \subsection{Sieci typu GAN}
  {
    GAN
  }
  \subsection{Generowanie obrazów rąk}
  {
    generowanie
  }
  \subsection{GestureGAN}
  {
    GestureGAN
  }
  \subsection{PoseGAN}
  {
    PoseGAN
  }
}

\section{Narzędzia i technologie wybrane do realizacji projektu}
{
  \subsection{Python}
  {
    python
  }
  \subsection{PyTorch}
  {
    pytorch
  }
  \subsection{MediaPipe}
  {
    mediapipe
  
  }
  \subsection{OpenCV}
  {
    opencv
  }
}

\section{Proces tworzenia projektu}
{
  \subsection{Dobór zbioru danych}
  {
    zbior
  }
  \subsection{Wybór architektury sieci}
  {
    architektura
  }
  \subsection{Preprocesowanie danych}
  {
    Preprocesowanie
  }
  \subsection{Tworzenie modelu}
  {
    model
  }
  \subsection{Tesotowanie modelu}
  {
    Tesotowanie
  }
  \subsection{Realizacja projektu}
  {
    realizacja
  }
}

\section{Wyniki i dyskusja}
{
  wyniki
}

\section{Podsumowanie}
{
  \subsection{Zalety i wady przyjętych rozwiązań}
  {
    Zalety
  }
  \subsection{Napotkane trudności}
  {
    trudności
  }
  \subsection{Możliwości rozwoju}
  {
    rozwoj
  }
  \subsection{Wnioski końcowe}
  {
    wnioski
  }
}

\clearpage
\printbibliography[
  heading=bibintoc,
  title={Bibliografia}
]

\clearpage
\listoffigures

\clearpage
\listoftables

\clearpage
\addcontentsline{toc}{section}{Listingi}
\lstlistoflistings

\end{sloppypar}
\end{document}